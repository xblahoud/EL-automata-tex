% Automata stuff
\usetikzlibrary{automata}
\usetikzlibrary{arrows.meta}
\usetikzlibrary{bending}
\usetikzlibrary{shapes.callouts}
\usetikzlibrary{quotes}
\usetikzlibrary{positioning}
\usetikzlibrary{calc}
\usetikzlibrary{extshapes}
\usetikzlibrary{matrix}

\def\accsize{6pt}

\tikzset{
  >={Stealth[round,bend]},
}

\tikzstyle{automaton}=[
  % Use nice arrows that do not touch their destination.
  semithick,shorten >=1pt,>={Stealth[round,bend]},
  % Encourage a common distance between all states.
  node distance=1.5cm,
  % Disable the "start" text in front of the initial arrow.
  initial text=,
  % Allows to use scale to scale the figure
%  transform shape,
  % Reduce the size of the hidden node at the beginning of the initial arrow.
  every initial by arrow/.style={every node/.style={inner sep=0pt}},
  % Encourage a common size of all states that is smaller than the default.
  every state/.style={
    align=center,
    fill=white,
    minimum size=7.5mm,
    inner sep=0pt,
    % all states have the same height and baseline
    execute at begin node=\strut,
  },%
]

\tikzstyle{smallautomaton}=[
  automaton,
  node distance=8mm,
  every state/.style={minimum size=4mm,fill=white,inner sep=1pt}
]

\tikzstyle{mediumautomaton}=[
  automaton,
  node distance=1cm,
  every state/.style={minimum size=5mm,fill=white,inner sep=2pt}
]

\tikzstyle{cstate}=[state,capsule,text width=,inner xsep=-5pt]
\tikzstyle{dot}=[fill=black,circle,minimum size=4pt,inner sep=0]

\makeatletter
\tikzoption{initial angle}{\tikzaddafternodepathoption{\def\tikz@initial@angle{#1}}}
\makeatother

% pass "initial overlay" as a parameter to the tikz figure if you want
% to initial arrow to be ignored from the computation of the figure
% box.  This might be needed for better caption centering.
\tikzstyle{initial overlay}=[every initial by arrow/.append style={overlay}]

\tikzstyle{unreachable} = [densely dotted]

% % % Usage: \node[acclabel]{label\strut}
% The command \strut makes all labels have the same height.
\tikzstyle{acclabel} = [
  fill=yellow!20,
  rounded corners=3pt,
  draw=black,
  inner ysep=0pt,
  inner xsep=3pt,
]
\tikzstyle{ltllabel} = [acclabel,fill=darkgreen!20]

\tikzstyle{namelabel} = [
  execute at begin node = {$($},%
  execute at end node = {$)$}
]

\tikzstyle{matrix of states} = [
matrix of nodes,
every node/.style={state},
column sep=.8cm,
row sep=.8cm,
execute at empty cell={
  \node[transparent,name=
    \tikzmatrixname-\the\pgfmatrixcurrentrow-\the\pgfmatrixcurrentcolumn]{};]},
]

%\tikzset{
%  collacc0/.style={fill=blue!50!cyan},
%  collacc1/.style={fill=magenta},
%  collacc2/.style={fill=orange!90!black},
%  collacc3/.style={fill=green!70!black},
%  collacc4/.style={fill=blue!50!black},
%}

% Acceptance sets are displayed as small circles with white
% sans-serif number.  The default color is light blue. Can be
% changed by collacc styles.  Use anchor=center to ignore the
% predefined anchor on edges like "loop above" or "loop left"
% or when using path[auto].
\tikzstyle{accset}=[
  circle,inner sep=.9pt,
  draw=none, %draw=white,
  solid,
  collacc0, text=white,
  thin,
  anchor=center,
  font=\bfseries\tiny, %\sffamily
  minimum size={\accsize},
]
% Acceptance set as square
\tikzstyle{accsquare}=[accset,rectangle,inner sep=1.9pt,rounded corners=0pt]

% % % Colors % % %
\tikzset{
  collacc0/.style={fill=blue!50!cyan},
  collacc1/.style={fill=orange!90!black},
  collacc2/.style={fill=green!70!black},
  collacc3/.style={fill=red!80!black},
  collacc4/.style={fill=magenta},
  collacc5/.style={fill=blue!50!black},
}


% The key /sacc/where is used to determine where the acc marks on
% states should be placed using the sacc/derived keys
% The default value is center
% It has to be used in scopes, does not work inside nodes
\pgfkeyssetvalue{/sacc/where}{center}
\tikzset{
  sacc where/.code={
    \pgfkeyssetvalue{/sacc/where}{#1}
  }
}

% % % Labels and marks of automata % % %
\tikzset{ 
  % Labels of edges that have some acceptance mark. Shifts label
  % further from the edge to avoid ugly clash with the mark.
  % Usage: 
  %   (n1) edge pic[pos=.3,left]{l=LABEL}
  l/.pic={\node[outer sep=2pt] {#1};},%
  % A colored mark with number inside. Use acc for a small circle and
  % accsq for a small square.
  % Usage:
  %   (n1) edge pic[pos=.3]{acc=ACC_ID}
  % collaccACC_ID must be defined
  acc/.pic={\node[accset,collacc#1]{#1};},%
  accsq/.pic={\node[accsquare,collacc#1]{#1};},%
  % No number inside the bullets
  eacc/.pic={\node[accset,collacc#1]{\emptyacc};},%
  eaccsq/.pic={\node[accsquare,collacc#1]{\emptyacc};},%
  % State labels on north east
  sacc/.style = {
    append after command=
      {pic at (\tikzlastnode.\pgfkeysvalueof{/sacc/where}) {acc=#1}}
  },
  saccsq/.style = {
    append after command=
      {pic at (\tikzlastnode.\pgfkeysvalueof{/sacc/where}) {accsq=#1}}
  },
  esacc/.style = {
    append after command=
      {pic at (\tikzlastnode.\pgfkeysvalueof{/sacc/where}) {eacc=#1}}
  },
  esaccsq/.style = {
    append after command=
      {pic at (\tikzlastnode.\pgfkeysvalueof{/sacc/where}) {eaccsq=#1}}
  },
  % Easy typesetting of colored accepting marks with custom labels.
  % Usage:
  %   (n1) edge pic[pos=.3]{acc={COLOR_ID}{LABEL}}
  % collaccCOLOR_ID must be defined
  pics/cacc/.style 2 args={%
    code={\node[accset,collacc#1]{#2};}%
  },%
  pics/caccsq/.style 2 args={%
      code={\node[accsquare,collacc#1]{#2};}%
    },%
  csacc/.style 2 args = {%
      append after command=%
        {pic at (\tikzlastnode.\pgfkeysvalueof{/sacc/where}) {cacc={#1}{#2}}}%
  },%
  csaccsq/.style 2 args = {%
      append after command=%
        {pic at (\tikzlastnode.\pgfkeysvalueof{/sacc/where}) {caccsq={#1}{#2}}}%
  },%
}

% Macros for colored acceptance marks used in text. Usage:
% \tacc{COLOR_ID}{LABEL}
\def\markbaseline{-.37em}
\def\tacc#1{\tikz[baseline=\markbaseline]\pic{acc=#1};\xspace}
\def\tcacc#1#2{\tikz[baseline=\markbaseline]\pic{cacc={#1}{#2}};\xspace}
\def\taccsq#1{\tikz[baseline=\markbaseline]\pic{accsq=#1};\xspace}
\def\tcaccsq#1#2{\tikz[baseline=\markbaseline]\pic{caccsq={#1}{#2}};\xspace}

% % % some predefined marks with no number inside
\def\emptyacc{\phantom{0}}
\def\gm{\tcacc{2}{\emptyacc}}
\def\om{\tcacc{1}{\emptyacc}}
\def\bm{\tcacc{0}{\emptyacc}}
\def\bsq{\tcaccsq{0}{\emptyacc}}
\def\rsq{\tcaccsq{3}{\emptyacc}}

\def\minimark#1{
\def\emptyacc{}
\def\accsize{4pt}
#1
}

%\tikzset{
%  pics/accsq/.style 2 args={
%      code={ %
%        \node[accsquare,collacc#1]{#2};
%        }
%    },
%  % Construction for labels of edges that have some acceptance
%  % mark. Shifts label further from the edge to avoid ugly clash
%  % with the mark. Usage: pic[pos=.3,left]{l=LABEL}
%  pics/l/.style={
%      code={ %
%        \node[outer sep=3pt] {#1};
%        }
%    }
%}

% Macros for colored acceptance marks used in text. Usage:
% \tacc{COLOR_ID}{LABEL}
%\def\tacc#1#2{\tikz[smallautomaton,baseline=-3pt] \pic{acc={#1}{#2}};}
%\def\taccsq#1#2{\tikz[smallautomaton,baseline=-3pt] \pic{accsq={#1}{#2}};}